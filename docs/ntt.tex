\documentclass[twocolumn, uplatex, dvipdfmx]{jsarticle}

% 参照確認用
% \usepackage{refcheck}

% AMS系,定理環境,太字
\usepackage{amsmath, amssymb, amsthm}
\usepackage{bm}

% フォント・スクリプト文字
\usepackage{newtxtext, newtxmath}
\usepackage{mathrsfs}

% 図・表・箇条書き
\usepackage{tikz}
\usepackage{booktabs}
\usepackage{enumerate}
\usepackage{graphicx}

% アルゴリズム
\usepackage{algorithm, algorithmic}

% 索引
\usepackage{makeidx}

% その他
\usepackage{url, color}

% 不等号のデザイン変更
\renewcommand{\ge}{\geqslant}
\renewcommand{\le}{\leqslant}

% 数式環境の改行を許す.
\allowdisplaybreaks[4]

% 数式番号を "(節番号.数式番号)" とする.
\numberwithin{equation}{section}

% jsarticleのための付録の表示調整
%
% 付録は以下のように書く
% > \section{ ...通常の節... }
% > ......
% > \section{ ...通常の節... }
% >
% > % 付録開始
% > \appendix
% > \AppendixSection{ ...付録の節... }
% > ......
% > \AppendixSection{ ...付録の節... }
\newcommand{\AppendixSection}[1]{
\def\thesection{付録\Alph{section}} %
\section{#1} %
\def\thesection{\Alph{section}}}

% 定理環境設定
\theoremstyle{definition}

\newtheorem{prop}{Proposition}[section]
\newtheorem{thm}{Theorem}[section]
\newtheorem{lem}{Lemma}[section]

\begin{document}

\section{Number theoretic transform の導出}

$n,m$ を自然数とし,$N=2^m$ とする.
$R=\mathbb{Z}/n\mathbb{Z}$ は $\omega_N^N=1$ を満たす $\omega_N\in R$ をもつとする.
$(f_k)_{k=0,1,\dots,N-1}$ を $R$ の点列とし,
\begin{align}
	F_k=\sum_{j=0}^{N-1}\omega_N^{jk}f_j,\quad k=0,1,\dots,N-1\label{eq:F_k}
\end{align}
とする.
本稿では,$(f_k)$ から $(F_k)$ の効率的な計算方法について述べる.
素朴に計算をすれば,$\omega_N^{jk}$ の計算が $O(1)$ であってさえ,全体の計算量は $O(N^2)$ である.

\begin{lem}\label{lem:omega_exists}
	$n,m$ を自然数とし,$N=2^m$ とする.
	$R=\mathbb{Z}/n\mathbb{Z}$ は $\omega_N^N=1$ を満たす $\omega_N\in R$ をもつとする.
	このとき,$0\le l\le m$ について,$\omega_{N/2^l}^{N/2^l}=1$ を満たす $\omega_{N/2^l}\in R$ が存在する.
\end{lem}
\begin{proof}
	$0\le l\le m$ とする.$(\omega_N^{2^{m-l}})^{2^l}=\omega_N^{2^m}=\omega_N^N=1$ より,$\omega_{N/2^l}=\omega_N^{2^{m-l}}\in R$ である.
\end{proof}

\begin{lem}\label{lem:omega_square}
	$l=0,1,\dots,m-1$ について,$\omega_{N/2^l}^2=\omega_{N/2^{l+1}}$ である.
\end{lem}
\begin{proof}
	$l=0,1,\dots,m-1$ とする.$(\omega_{N/2^l}^2)^{N/2^{l+1}}=\omega_{N/2^l}^{N/2^l}=1$ である.
\end{proof}

\begin{lem}\label{lem:omega_-1}
	$l=0,1,\dots,m-1$ について,$\omega_{N/2^l}^{N/2^{l+1}}=-1$ である.
\end{lem}
\begin{proof}
	$l=0,1,\dots,m-1$ とする.$(\omega_{N/2^l}^{N/2^{l+1}})^2=\omega_{N/2^l}^{N/2^l}=1$ である.
	よって,$\omega_{N/2^l}^{N/2^{l+1}}=-1$ である.
\end{proof}

\begin{prop}\label{prop:eo}
	$0\le k\le N/2-1$ とする.
	$F_k^e$, $F_k^o$ を,
	\begin{align}
		&F_k^e=\sum_{j=0}^{N/2-1}\omega_{N/2}^{jk}f_{2j},\\
		&F_k^o=\sum_{j=0}^{N/2-1}\omega_{N/2}^{jk}f_{2j+1}
	\end{align}
	とすると,
	\begin{align}
		&F_k=F_k^e+\omega^kF_k^o,\\
		&F_{k+N/2}=F_k^e-\omega^kF_k^o,
	\end{align}
	が成り立つ.
\end{prop}
\begin{proof}
	\begin{align*}
		F_k&=\sum_{j=0}^{N-1}\omega_N^{jk}f_j\\
		&=\sum_{j=0,2,\dots,N-2}\omega_N^{jk}f_j+\sum_{j=1,3,\dots,N-1}\omega_N^{jk}f_j\\
		&=\sum_{j=0}^{N/2-1}\omega_N^{2jk}f_{2j}+\sum_{j=0}^{N/2-1}\omega_N^{(2j+1)k}f_{2j+1}\\
		&=\sum_{j=0}^{N/2-1}(\omega_N^2)^{jk}f_{2j}+\sum_{j=0}^{N/2-1}\omega_N^k(\omega_N^2)^{jk}f_{2j+1}
	\end{align*}
	であるが,補題\ref{lem:omega_square}より,
	\begin{align*}
		F_k&=\sum_{j=0}^{N-1}\omega_N^{jk}f_j\notag\\
		&=\sum_{j=0}^{N/2-1}\omega_{N/2}^{jk}f_{2j}+\omega_N^k\sum_{j=0}^{N/2-1}\omega_{N/2}^{jk}f_{2j+1}\\
		&=F_k^e+\omega^kF_k^o
	\end{align*}
	である.
	また,
	\begin{align*}
		&F_{k+N/2}\\
		&=\sum_{j=0}^{N-1}\omega_N^{j(k+N/2)}f_j\notag\\
		&=\sum_{j=0}^{N/2-1}\omega_{N/2}^{j(k+N/2)}f_{2j}+\omega_N^{k+N/2}\sum_{j=0}^{N/2-1}\omega_{N/2}^{j(k+N/2)}f_{2j+1}
	\end{align*}
	において,$j=0,1,\dots,N/2-1$ について $\omega_{N/2}^{j(k+N/2)}=\omega_{N/2}^{jk}(\omega_{N/2}^{N/2})^k=\omega_{N/2}^{jk}$ であることと,$\omega_N^{k+N/2}=\omega_N^k\omega_N^{N/2}$ と補題\ref{lem:omega_-1}より,
	\begin{align*}
		&F_{k+N/2}\\
		&=\sum_{j=0}^{N/2-1}\omega_{N/2}^{jk}f_{2j}-\omega_N^k\sum_{j=0}^{N/2-1}\omega_{N/2}^{jk}f_{2j+1}\\
		&=F_k^e-\omega^kF_k^o
	\end{align*}
	である.
\end{proof}

\begin{thm}
	$1\le l\le m$ とし,$b_i\in\{e,o\}$, $i=0,1,\dots,l-1$ とする.
	$B:\{e,o\}\to\{0,1\}$ を,
	\begin{align}
		B(b)=\begin{cases}
			0,&\quad b=e,\\
			1,&\quad b=o
		\end{cases}
	\end{align}
	で定める.
	$F_k^{b_0b_1\cdots b_{l-1}}$ を
	\begin{align}
		F_k^{b_0b_1\cdots b_{l-1}}=\sum_{j=0}^{N/2^l-1}f_{2^lj+\sum_{i=0}^{l-1}2^iB(b_i)}\omega_{N/2^l}^{jk}
	\end{align}
	とするとき,$0\le k<N-N/2^{l+1}$, $1\le l<m$ ならば,
	\begin{align}
		&F_k^{b_0b_1\cdots b_{l-1}}\notag\\
		&\quad=F_k^{b_0b_1\cdots b_{l-1}e}+\omega_{N/2^l}^kF_k^{b_0b_1\cdots b_{l-1}o},\\
		&F_{k+N/2^{l+1}}^{b_0b_1\cdots b_{l-1}}\notag\\
		&\quad=F_k^{b_0b_1\cdots b_{l-1}e}-\omega_{N/2^l}^kF_k^{b_0b_1\cdots b_{l-1}o}
	\end{align}
	が成り立つ.
\end{thm}
\begin{proof}
	$0\le l<m$ とする.
	まず,
	\begin{align*}
		&F_k^{b_0b_1\cdots b_{l-1}}\\
		&=\sum_{j=0}^{N/2^l-1}f_{2^lj+\sum_{i=0}^{l-1}2^iB(b_i)}\omega_{N/2^l}^{jk}\\
		&=\sum_{j=0,2,\dots,N/2^l-2}f_{2^lj+\sum_{i=0}^{l-1}2^iB(b_i)}\omega_{N/2^l}^{jk}\\
		&\quad+\sum_{j=1,3,\dots,N/2^l-1}f_{2^lj+\sum_{i=0}^{l-1}2^iB(b_i)}\omega_{N/2^l}^{jk}\\
		&=\sum_{j=0}^{N/2^{l+1}-1}f_{2^l\cdot2j+\sum_{i=0}^{l-1}2^iB(b_i)}\omega_{N/2^l}^{2jk}\\
		&\quad+\sum_{j=0}^{N/2^{l+1}-1}f_{2^l(2j+1)+\sum_{i=0}^{l-1}2^iB(b_i)}\omega_{N/2^l}^{(2j+1)k}\\
		&=\sum_{j=0}^{N/2^{l+1}-1}f_{2^{l+1}j+2^lB(e)+\sum_{i=0}^{l-1}2^iB(b_i)}\omega_{N/2^{l+1}}^{jk}\\
		&\quad+\omega_{N/2^l}^k\sum_{j=0}^{N/2^{l+1}-1}f_{2^{l+1}j+2^lB(o)+\sum_{i=0}^{l-1}2^iB(b_i)}\omega_{N/2^{l+1}}^{jk}\\
		&=F_k^{b_0b_1\cdots b_{l-1}e}+\omega_{N/2^l}^kF_k^{b_0b_1\cdots b_{l-1}o}
	\end{align*}
	である.
	一方,
	\begin{align*}
		&F_{k+N/2^{l+1}}^{b_0b_1\cdots b_{l-1}}\\
		&=\sum_{j=0}^{N/2^{l+1}-1}f_{2^{l+1}j+2^lB(e)+\sum_{i=0}^{l-1}2^iB(b_i)}\omega_{N/2^{l+1}}^{j(k+N/2^{l+1})}\\
		&\quad+\omega_{N/2^l}^{k+N/2^{l+1}}\\
		&\quad\quad\times\sum_{j=0}^{N/2^{l+1}-1}f_{2^{l+1}j+2^lB(o)+\sum_{i=0}^{l-1}2^iB(b_i)}\omega_{N/2^{l+1}}^{j(k+N/2^{l+1})}
	\end{align*}
	であるが,$\omega_{N/2^{l+1}}^{k+N/2^{l+1}}=\omega_{N/2^{l+1}}^k\omega_{N/2^{l+1}}^{N/2^{l+1}}=\omega_{N/2^{l+1}}^k$ であることと,$\omega_{N/2^l}^{k+N/2^{l+1}}=\omega_{N/2^l}^k\omega_{N/2^l}^{N/2^{l+1}}$ と補題\ref{lem:omega_-1}より,
	\begin{align*}
		&F_{k+N/2^{l+1}}^{b_0b_1\cdots b_{l-1}}\\
		&=\sum_{j=0}^{N/2^{l+1}-1}f_{2^{l+1}j+2^lB(e)+\sum_{i=0}^{l-1}2^iB(b_i)}\omega_{N/2^{l+1}}^{jk}\\
		&\quad-\omega_{N/2^l}^k\\
		&\quad\quad\times\sum_{j=0}^{N/2^{l+1}-1}f_{2^{l+1}j+2^lB(o)+\sum_{i=0}^{l-1}2^iB(b_i)}\omega_{N/2^{l+1}}^{jk}\\
		&=F_k^{b_0b_1\cdots b_{l-1}e}-\omega_{N/2^l}^kF_k^{b_0b_1\cdots b_{l-1}o}
	\end{align*}
	である.
\end{proof}

\begin{lem}
	$0\le k\le N-1$ について,
	$F_k^{b_0b_1\cdots b_{m-1}}=f_{\sum_{j=0}^{m-1}2^jB(b_j)}$ である.
\end{lem}
\begin{proof}
	定義より,
	\begin{align*}
		F_k^{b_0b_1\cdots b_{m-1}}&=\sum_{j=0}^0f_{\sum_{i=0}^{m-1}2^iB(b_i)}\omega_{N/2^m}^{jk}\\
		&=f_{\sum_{j=0}^{m-1}2^jB(b_j)}
	\end{align*}
	である.
\end{proof}

\begin{thm}
	$\mathcal{B}=B^{-1}$ とする.
	$R_m:\mathbb{N}\to\mathbb{N}$ を,$k=\sum_{j=0}^{m-1}2^ja_j$, $a_j\in\{0,1\}$, $0\le j\le m-1$ に対して,$R_m(k)=\sum_{j=0}^{m-1}2^{m-1-j}a_j$ と定める写像とする.
	\begin{align}
		&F_k^{(0)}=f_{R_m(k)},\quad k=0,1,\dots,2^m-1,\\
		&F_k^{(m)}=F_k,\quad k=0,1,\dots,2^m-1
	\end{align}
	とする.
	$0\le k\le 2^m-1$, $1\le l\le m-1$ について,$k=2^lq+r$, $0\le r<2^l$,
	\begin{align}
		q=\sum_{j=0}^{m-l-1}2^ja'_j,\quad a'_j\in\{0,1\},\quad 0\le j\le m-l-1
	\end{align}
	として,
	\begin{align}
		F_k^{(l)}=F_r^{\mathcal{B}(a'_{m-l-1})\mathcal{B}(a'_{m-l-2})\cdots\mathcal{B}(a'_0)}
	\end{align}
	とする.
	このとき,$l=1,2,\dots,m$ について,$k=2^lq+r$, $r=0,1,\dots,2^{l-1}-1$, $q=0,1,\dots,2^{m-l}-1$ において,
	\begin{align}
		&F_k^{(l)}=F_k^{(l-1)}+\omega_{N/2^{m-l}}^rF_{k+2^{l-1}}^{(l-1)},\\
		&F_{k+2^{l-1}}^{(l)}=F_k^{(l-1)}-\omega_{N/2^{m-l}}^rF_{k+2^{l-1}}^{(l-1)}
	\end{align}
	が成り立つ.
\end{thm}
\begin{proof}
	$l=1$ のときを示す.$k=2q$, $0\le q\le 2^{m-2}$ とする.
	$k=\sum_{j=0}^{m-1}2^ja_j$ とすると,$a_0=0$ であるため,$k=\sum_{j=1}^{m-1}2^ja_j+0$, $k+1=\sum_{j=1}^{m-1}2^ja_j+1$ であり,$R_m(k+1)=R_m(k)+2^{m-1}=R_m(k)+N/2$ である.
	よって,$F_k^{(0)}=f_{R_m(k)}$ および,
	\begin{align*}
		\omega_{N/2^{m-1}}^kF_{k+1}^{(0)}&=f_{R_m(k)}+\omega_{N/2^{m-1}}^kf_{R_m(k+1)}\\
		&=\omega_{N/2^{m-1}}^kf_{R_m(k)+N/2}\\
		&=\omega_2^{2q}f_{R_m(k)+N/2}\\
		&=(\omega_2^2)^qf_{R_m(k)+N/2}\\
		&=f_{R_m(k)+N/2}
	\end{align*}
	である.
	一方,$k=2q$, $q=\sum_{j=0}^{m-2}2^ja'_j$ とすると,$a'_j=a_{j+1}$ なので,$\sum_{i=0}^{m-2}2^{m-2-i}a'_i=\sum_{i=0}^{m-2}2^{m-2-i}a_{i+1}=\sum_{i=1}^{m-1}2^{m-1-i}a_i=R_m(k)$ より,
	\begin{align*}
		F_k^{(1)}&=F_0^{\mathcal{B}(a'_{m-2})\mathcal{B}(a'_{m-3})\cdots\mathcal{B}(a'_0)}\\
		&=f_{\sum_{i=0}^{m-2}2^{m-2-i}a'_i}+\omega_{N/2^{m-1}}^0f_{2^{m-1}+\sum_{i=0}^{m-2}2^{m--2-i}a'_i}\\
		&=f_{\sum_{i=0}^{m-2}2^{m-2-i}a'_i}+f_{2^{m-1}+\sum_{i=0}^{m-2}2^{m-2-i}a'_i}\\
		&=f_{R_m(k)}+f_{N/2+R_m(k)}\\
		&=F_k^{(0)}+\omega_{N/2^{m-1}}^kF_{k+1}^{(0)}
	\end{align*}
	によって成り立つ.
	また,
	\begin{align*}
		F_{k+1}^{(l)}&=F_1^{\mathcal{B}(a'_{m-2})\mathcal{B}(a'_{m-3})\cdots\mathcal{B}(a'_0)}\\
		&=f_{\sum_{i=0}^{m-2}2^{m-2-i}a'_i}+\omega_{N/2^{m-1}}^1f_{2^{m-1}+\sum_{i=0}^{m-2}2^{m-2-i}a'_i}\\
		&=f_{\sum_{i=0}^{m-2}2^{m-2-i}a'_i}+\omega_2^1f_{2^{m-1}+\sum_{i=0}^{m-2}2^{m-2-i}a'_i}\\
		&=f_{R_m(k)}-f_{N/2+R_m(k)}\\
		&=F_k^{(0)}-\omega_{N/2^{m-1}}^kF_{k+1}^{(0)}
	\end{align*}
	である.

	次に,$2\le l\le m-1$ とする.
	$k=2^lq+r$, $q=\sum_{j=0}^{m-l-1}2^ja'_j$ とおく.
	このとき,
	\begin{align*}
		F_k^{(l)}&=F_r^{\mathcal{B}(a'_{m-l-1})\mathcal{B}(a'_{m-l-2})\cdots\mathcal{B}(a'_0)}\\
		&=F_r^{\mathcal{B}(a'_{m-l-1})\mathcal{B}(a'_{m-l-2})\cdots\mathcal{B}(a'_0)e}\\
		&\quad+\omega_{N/2^l}^k\\
		&\quad\quad\times F_r^{\mathcal{B}(a'_{m-l-1})\mathcal{B}(a'_{m-l-2})\cdots\mathcal{B}(a'_0)o}
	\end{align*}
	である.
	また,$k=2^{l-1}\hat{q}+\hat{r}$, $\hat{q}=\sum_{j=0}^{m-l}2^j\hat{a}'_j$ とおくと,
	$\hat{a}'_j=0$, $\hat{a}'_j=a'_{j-1}$, $\hat{r}=r$ より,
	\begin{align*}
		F_k^{(l-1)}&=F_r^{\mathcal{B}(\hat{a}'_{m-l})\mathcal{B}(\hat{a}'_{m-l-1})\cdots\mathcal{B}(\hat{a}'_0)}\\
		&=F_r^{\mathcal{B}(\hat{a}'_{m-l})\mathcal{B}(\hat{a}'_{m-l-1})\cdots\mathcal{B}(\hat{a}'_1)e}\\
		&=F_r^{\mathcal{B}(a'_{m-l-1})\mathcal{B}(a'_{m-l-2})\cdots\mathcal{B}(a'_0)e}
	\end{align*}
	かつ,$k+2^{l-1}=2^{l-1}(\hat{q}+1)+\hat{r}$ より,
	\begin{align*}
		F_{k+2^{l-1}}^{(l-1)}&=F_r^{\mathcal{B}(\hat{a}'_{m-l})\mathcal{B}(\hat{a}'_{m-l-1})\cdots\mathcal{B}(\hat{a}'_1)o}\\
		&=F_r^{\mathcal{B}(a'_{m-l-1})\mathcal{B}(a'_{m-l-2})\cdots\mathcal{B}(a'_0)o}
	\end{align*}
	であるため,
	\begin{align*}
		F_k^{(l)}=F_k^{(l-1)}+\omega_{N/2^{m-l}}^kF_{k+2^{l-1}}^{(l-1)}
	\end{align*}
	が成り立つ.
	
	一方,
	\begin{align*}
		F_{k+2^{l-1}}^{(l)}&=F_{r+2^{l-1}}^{\mathcal{B}(a'_{m-l-1})\mathcal{B}(a'_{m-l-2})\cdots\mathcal{B}(a'_0)}\\
		&=F_{r+N/2^{m-l+1}}^{\mathcal{B}(a'_{m-l-1})\mathcal{B}(a'_{m-l-2})\cdots\mathcal{B}(a'_0)}\\
		&=F_r^{\mathcal{B}(a'_{m-l-1})\mathcal{B}(a'_{m-l-2})\cdots\mathcal{B}(a'_0)e}\\
		&\quad-\omega_{N/2^{m-l}}^k\\
		&\quad\quad\times F_r^{\mathcal{B}(a'_{m-l-1})\mathcal{B}(a'_{m-l-2})\cdots\mathcal{B}(a'_0)o}\\
		&=F_k^{(l-1)}-\omega_{N/2^{m-l}}^kF_{k+2^{l-1}}^{(l-1)}
	\end{align*}
	である.


	最後に,$l=m$ とする.
	$k=0,1,\dots,2^{m-1}-1$ とする.
	$k=2^{m-1}0+k$ より,
	\begin{align*}
		&F_k^{(m-1)}+\omega_N^kF_{k+2^{m-1}}^{(m-1)}\\
		&=F_k^{\mathcal{B}(0)}+\omega_N^kF_k^{\mathcal{B}(1)}\\
		&=F_k^e+\omega_N^kF_k^o\\
		&=F_k\\
		&=F_k^{(m)}
	\end{align*}
	であり,
	\begin{align*}
		&F_{k+2^{m-1}}^{(m)}\\
		&=F_{k+2^{m-1}}\\
		&=F_k^{\mathcal{B}(0)}-\omega_N^kF_k^{\mathcal{B}(1)}\\
		&=F_k^e-\omega_N^kF_k^o\\
		&=F_k^{(m-1)}-\omega_N^kF_{k+2^{m-1}}^{(m-1)}
	\end{align*}
	である.

	以上より,すべての $l=1,2,\dots,m$ について,主張が示された.
\end{proof}

\begin{thm}\label{thm:ntt}
	以下の手順で,$(f_k)$ から $(F_k)$ が求まる.
	ただし,$a\leftarrow b$ で,$a$ の値を $b$ で上書きすることを表す.
	\begin{enumerate}
		\item $(f_0,f_1,\dots,f_{N-1})$\\
			\quad$\leftarrow (f_{R_m(0)},f_{R_m(1)},\dots,f_{R_m(N-1)})$
		\item $(f_k,f_{k+2^l})$\\
			\quad$\leftarrow (f_k+\omega_{N/2^{m-l}}^rf_{k+2^{l-1}},f_k-\omega_{N/2^{m-l}}^rf_{k+2^{l-1}})$,\\
			$k=2^lq+r$,\quad $r=0,1,\dots,2^{l-1}-1$,\\
			$q=0,1,\dots,2^{m-l}-1$,\quad $l=1,2,\dots,m$
		\item $(F_0,F_1,\dots,F_{N-1})\leftarrow (f_0,f_1,\dots,f_{N-1})$.
	\end{enumerate}
\end{thm}
\begin{proof}
	これまでのの議論から明らか.
\end{proof}

\begin{prop}
	定理\ref{thm:ntt}の演算量は,項番1の演算量が $O(N\log_2N)$ であり,$\omega_{N/2^{m-l}}^k$ の演算量が $O(\log_2N)$ であれば,$O(N\log_2N)$ である.
\end{prop}
\begin{proof}
	項番2の反復回数が $O(N\log_2N)$ なので明らか.
\end{proof}

\end{document}
